%%%%%%%%%%%%%%%
% This CV example/template is based on my own
% CV which I (lamely attempted) to clean up, so that
% it's less of an eyesore and easier for others to use.
%
% LianTze Lim (liantze@gmail.com)
% 13 May, 2020
%
\documentclass[a4paper,skipsamekey,11pt,english]{curve}

% Uncomment to enable Chinese; needs XeLaTeX
% \usepackage{ctex}

\usepackage{settings}

% Change the fonts if you want
\ifxetexorluatex % If you're using XeLaTeX or LuaLaTeX
  \usepackage{fontspec} 
  %% You can use \setmainfont etc; I'm just using these font packages here because they provide OpenType fonts for use by XeLaTeX/LuaLaTeX anyway
  \usepackage[p,osf,swashQ]{cochineal}
  \usepackage[medium,bold]{cabin}
  \usepackage[varqu,varl,scale=0.9]{zi4}
\else % If you're using pdfLaTeX or latex
  \usepackage[T1]{fontenc}
  \usepackage[p,osf,swashQ]{cochineal}
  \usepackage{cabin}
  \usepackage[varqu,varl,scale=0.9]{zi4}
\fi

%% Only needed if you want a Publication List
\addbibresource{own-bib.bib}

%% Specify your last name and first name (as given in the .bib) to automatically bold your own name in the publications list. One caveat: You need to write \bibnamedelima where there's a space in your name for this to work properly for now...
\myname{Lim}{Lian\bibnamedelima Tze}
% \myname{d'Andrimont}{Raphaël}

% Change the page margins if you want
% \geometry{left=1cm,right=1cm,top=1.5cm,bottom=1.5cm}

% Change the colours if you want
% \definecolor{SwishLineColour}{HTML}{00FFFF}
% \definecolor{MarkerColour}{HTML}{0000CC}

% Change the item prefix marker if you want
% \prefixmarker{$\diamond$}

%% Photo is only shown if "fullonly" is included
\includecomment{fullonly}
% \excludecomment{fullonly}


%%%%%%%%%%%%%%%%%%%%%%%%%%%%%%%%%%%%%%


\leftheader{%
  {\LARGE\bfseries\sffamily Manuel S\'anchez del R{\'i}o, Ph.D.}

  \makefield{\faEnvelope[regular]}{\texttt{srio@esrf.eu}} \\
  \makefield{}{\texttt{Date of birth 1963-07-27}} \\
  \makefield{}{\texttt{Home address: 7 Avenue F\'elix Viallet 38000 Grenoble (France)}} \\
  \makefield{}{\texttt{Phone: +33 677 40 80 82}} \\
  
  %\makefield{\faTwitter}{\texttt{@example}}

  %\makefield{\faGlobe}{\url{http://example.example.org/}}

  %\makefield{\faLinkedin}
%   {\url{http://www.linkedin.com/in/example/}}
}

\rightheader{~}
\begin{fullonly}
\photo[r]{photo}
\photoscale{0.13}
\end{fullonly}

\title{Curriculum Vitae}

\begin{document}
\makeheaders[c]

\makerubric{education}

\makerubric{employment}

\begin{rubric}{Overview of Research}
\end{rubric}

% If you're not a researcher nor an academic, you probably don't have any publications; delete this line.
%% Sometimes when a section can't be nicely modelled with the \entry[]... mechanism; hack our own and use \input NOT \makerubric
%% Sometimes when a section can't be nicely modelled with the \entry[]... mechanism; hack our own
\makerubrichead{Selected publications}

%% Assuming you've already given \addbibresource{own-bib.bib} in the main doc. Right? Right???
\nocite{*}


I have over 150 research papers in peer-reviewed international research journals:
\begin{itemize}
    \item The full list of publications is found at \makefield{\faLink}{\url{http://publicationslist.org/srio}}
    \item 119
    %160-41
    Journal publications, 42
    % 54-12
    as first author
    \item plus 41 SPIE publications, 12 as first author
    % \item XX Book chapters, XX as first author
    % \item XX Conference presentations
    \item Citations: 3996  •  h-index: 34  •  i10-index: 81  •  Google Scholar (as of May. 16, 2021)
\end{itemize}


% If you just want everything in one list
\printbibliography[heading={none}]
% \printbibliography[heading={none},keyword={selected}]


% \printbibliography[heading={subbibliography},title={Journal Articles},type=article]

% \printbibliography[heading={subbibliography},title={Conference Proceedings},type=inproceedings]

% \printbibliography[heading={subbibliography},title={Books and Chapters},filter={booksandchapters}]

% \printbibliography[heading=subbibliography]

\makerubric{skills}
\makerubric{misc}

\makerubric{referee}
% %% Probably not the best way of doing it but what the heck, I just winged-it :p

\makerubrichead{References}

\begin{tabularx}{\textwidth}{@{}X X@{}}
\textbf{Prof X}\par
Professor\par
ABC University,\par 
Address.\par 
\makefield{\faEnvelopeO}{\url{abc@def.edu}}
& 
\textbf{Prof Y}\par
Professor\par
ABC University,\par 
Address.\par 
\makefield{\faEnvelopeO}{\url{abc@def.edu}}
\\
\end{tabularx}


\end{document}