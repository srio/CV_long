\begin{rubric}{Miscellaneous \inred{/ Services for the scientific community}}

\subrubric{Teaching}

\entry*[2021]Tutorial at the HERCULES European School "Modeling SR beamlines with OASYS" Grenoble, France (also in 2019 and 2018)
\entry*[2019/12]"Second OASYS School" APS-ANL, Lemont, IL, USA
\entry*[2019/05]"First OASYS School" ESRF, Grenoble, France
\entry*[2002]University post-graduate course "Aplicaciones de los rayos x en ciencia y tecnología" at the Universidad
Aut\'onoma Metropolitana, M\'exico City.
\entry*[2001]Short course "Computer simulations and ray-tracing for hard x-ray optics" at SPIE annual
meeting, San Diego CA.
\entry*[2001]"Hard x-ray optics simulations and ray tracing" at the Summer school of synchrotron radiation applications, Beijing China.
\entry*[1998]"Ray tracing" at the Summer school on synchrotron radiation: applications to materials science and physics Luso Portugal


% Ferrara (2001,1994), La Sapienza-Roma (1991)


\subrubric{Invited seminars}
%2018/10
\entry*[2018/10] DLSR2018 (The 6th Diffraction Limited Storage Ring) "Partial coherence in undulator beamlines at ultra-low emittance storage rings", Berkeley, CA, USA.
%2018-05 15 and 16
\entry*[2018/05] Diamond II Coherence Workshop "Partial coherence in undulator beamlines at ultra-low emittance storage rings", Abingdon, UK
%2018/03
\entry*[2018/03] FLS2018 (60th ICFA Advanced Beam Dynamics Workshop on
Future Light Sources) "Partial coherence in undulator beamlines at ultra-low emittance storage rings", Shanghai, China.

\entry*[2017] PHANGS2017 (Photons at the Next Generation Synchrotron Facilities) "Partial coherence in undulator beamlines at ultra-low emittance storage rings" Trieste Italy.
%2015/10/01
\entry*[2015] Meeting "Simulation and Modeling for SR Sources and X-Ray Optics” 2015/10 "Ray-tracing calculations for 2-3 pole wigglers" NSLS-II, Brookhaven National Laboratory, USA.
\entry*[2014] SPIE Optical Engineering + Applications "Simulations of X-ray optics for ESRF-Upgrade Programme", San Diego, USA.
%2014/11/25
\entry*[2014] APS Argonne USA "The evolution of X-ray optics simulations stimulated by the Upgrade Programme
"
%2013/10/20
\entry*[2013] Trieste MEADOW (MEtrology, Astronomy, Diagnostics and Optics Workshop) "The current situation of optical simulation codes for X-ray beamlines"
\entry*[2011] \inred{Cornell High Energy Synchrotron Source, USA}
\entry*[2011] ESRF Friday Lectures "Physics of X-ray radiation production and transport. Simulating photons and waves from the X-ray sources to the samples " Grenoble, France
%2011/02/17
\entry*[2011] Saint Gobain Paris "Algorithmes et logiciels pour la modelisation de la diffraction X sur des cristaux parfaits, mosaiques et multicouches"
\entry*[2010] NSLS Brookhaven USA "X-ray optics simulations and beamline modelling for the ESRF Upgrade Programme"
\entry*[2009] Diamond UK "X-ray optics simulations and beamline modelling for the ESRF Upgrade Programme"
\entry*[2006] Canadian Light Source Canada "Computer modeling of x-ray optics for synchrotron beams"


% \entry*[2006] Universit\'e Joseph Fourier Grenoble France (2006)
% \entry*[2005 -- 2002] UNAM Mexico
% \entry*[2001] Universit\`a di Ferrara, Italy
% \entry*[2002] APS Argonne USA
% \entry*[1998] SLS Switzerland
% \entry*[1996] ALBA Spain
% \entry*[1992] LURE France
% \entry*[1992] Frascati Italy
% \entry*[1991] Universit\`a La Sapienza Roma, Italy Italy

\subrubric{Organization of international conferences and events}
\entry*[2019/12] Lemont, IL, USA "Second OASYS School"
\entry*[2019/05] Grenoble, France "First OASYS School"
\entry*[2016] Trieste, Italy SOS "Software for Optical Simulations Workshop" 3-7 October 2016
\entry*[2014] San Diego, CA, USA SPIE (The International Society for Optical Engineering) conference 9209: "Advances in Computational methods for X-Ray Optics III"
%2013-06- 3-5
\entry*[2013] Grenoble, France "Synchrotron Optics Simulations: 3 codes Tutorial"
\entry*[2011] San Diego, CA, USA SPIE (The International Society for Optical Engineering) conference 8141: "Advances in Computational methods for X-Ray Optics II"
\entry*[2009] Grenoble, France SMEXOS "Simulation Methods for X-ray Optical Systems"
\entry*[2004] Denver, CO, USA SPIE (The International Society for Optical Engineering) conference 5536: "Advances in Computational methods for X-Ray and Neutron Optics".

\subrubric{Referee for scientific and technical journals}
\entry*[] Journal of Synchrotron Radiation, Review of Scientific Instrumentation, Nuclear Instruments and Methods,  Journal of Archaeological Science, Applied Clay Science, Microporous \& Mesoporous Materials, National Geographic, Journal of Physics and Chemistry of Solids, etc.

\subrubric{Supervision Ph.D. students}
\entry*[Rafael Celestre] "Investigations of the effect of optical imperfections on partially coherent X-ray beam by combining optical simulations with wavefront sensing experiments" Universit\'e Grenoble-Alpes (2021)

\entry*[Mark Glass] "Statistical optics for synchrotron emission: numerical calculation of coherent modes" Universit\'e Grenoble-Alpes (2017) 
% \makefield{\faLink}
{
\url{https://tel.archives-ouvertes.fr/tel-01664052}}

\entry*[Alvaro Martin-Ortega] "Power absorption mechanisms and energy transfer in X-ray gas attenuators" Universit\'e Grenoble-Alpes (2017) \url{https://tel.archives-ouvertes.fr/tel-01524361}

\entry*[Xianchao Cheng] "Thermal stress issues in thin film coatings of X-ray optics under high heat load" Universit\'e Grenoble-Alpes (2014) \url{https://tel.archives-ouvertes.fr/tel-01440220}

\entry*[Lucia Alianelli] "Characterization and Modelling of
Imperfect Crystals for Thermal Neutron Diffraction" Universit\'e Joseph Fourier (Grenoble) (2002)


% \entry*[Others] 6 master students
% % Davide Bianchi, Eleonora Secco, Sophie Thery, Giovanni Pirro, Edoardo Cappelli, Yunes
% and more than 10 Undergraduated students in their work as trainees at ESRF. 

% \subrubric{Attended courses}
% \entry*[]
% \begin{itemize}
% \item Especialista en Gestión de los Sistemas y las Tecnologías de la Información de la Empresa
% 1992-1994 Universidad Politécnica de Madrid, CEPADE (Centro de Estudios de Postgrado en Administración y Dirección de Empresas).
% \item Management course at ESRF (2001): Le management "dynargie".
% \item Courses on software languages and applications at ESRF (Phyton , Matlab, Mathematica,
% C++, IDL, Logiscope).
% \item Courses on Monte Carlo packages for Radiation Transport: EGS4 (Montpellier, June 10-13,
% 1996) and PENELOPE (Paris, Nov. 5-7, 2001).
% \item Simulación de Problemas Físicos mediante el método de los Elementos Finitos. Cursos de
% verano de Laredo (1994), Universidad de Cantabria.
% \item International School of Physics Enrico Fermi, "Photoemission and Absorption Spectroscopy
% of Solids and Interfaces with Synchrotron Radiation", (Varenna, Italy, July 12-22, 1988).
% \item CERN Accelerator School : General accelerator physics (Salamanca, Nov. 19 - 30, 1988). 
% \end{itemize}


\subrubric{Interests}
\entry*[] \inred{Application of analytical techniques to the study of art and archaeology. Ancient Mesoamerican cultures. Photography. Travel. Ancient cars. Classic and contemporary dance critic.
}
% \subrubric{Awards and Achievements}
% \entry*[2002] \textbf{Merit Award}, Random Training Course held at Secret Location.
% %
% \entry*[2001] \textbf{Department Prize for Outstanding Student Performance}, Unseen University.

% \subrubric{Certification}
% \entry*[2014] \textbf{Certified XYZ Practioner}. Awarded by X Insitute.
% \entry*[2006] \textbf{Certified Level 3 in ABC}. Awarded by ABC.

\end{rubric}