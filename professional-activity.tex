%!TEX encoding = UTF8
%!TEX root =cv-llt.tex



\prefix{}
\begin{rubric}{Professional activity}
% \entry*[]
%     My professional activity is devoted to X-ray instrumentation modelling and research using synchrotron light. It involves three main fields:
%     \begin{itemize}
%         \item Design, development, and use of computer codes in particular ray-tracing techniques for modelling X-ray optics with direct applications to the design of synchrotron radiation beamlines.
%         \item Development and use of data analysis tools 
%     specific for different synchrotron techniques (X-ray absorption spectroscopy, X-ray diffraction, X-ray fluorescence, etc.).
%         \item Experimental activity study of cultural heritage. Several projects concern the structure and chemistry of the Maya blue pigment.
%     \end{itemize}

\entry*[Summary]
    \textbf{My professional activity is devoted to X-ray instrumentation modelling and research using synchrotron light. I designed, developed, and applied computational codes, mainly ray-tracing and wave-optics, for modelling X-ray optics with direct applications to the design of synchrotron radiation beamlines.
    I contributed with new algorithms for accurate modeling most optical elements used with X-rays (crystals, mirrors, multilayers, etc.). I characterized experimentally some of these elements and used them for different synchrotron techniques (X-ray absorption spectroscopy, X-ray diffraction, etc.). I developed new methodologies for treating computationally the partial coherence of synchrotron beams. I am author of several computer codes used worlwide for x-ray optics simulations. The algorithms and tools I developped are used in many synchrotron facilities and other X-ray laboratories. I also developed experimental work for characterization of optical components. I gained international relevance in other fields, such as clay mineralogy and cultural heritage.}

% \end{rubric}

% \begin{rubric}{Professional activitydddddd}

\entry*[Detailed]

My professional activity is devoted to synchrotron radiation
instrumentation and research.

During my Ph.D. I worked at the Center for X-ray Lithography, University of Wisconsin-Madison.
I specialized in beamline simulations using ray-tracing, working with Prof. F. Cerrina (University of Wisconsin and Boston University), creator of SHADOW. I applied it to new beamline projects at the Laboratori Nazionali di Frascati-Istituto Nazionale di Fisica Nucleare.

I joined the ESRF at its construction phase in 1990 with the mission of developing and coordinating the optical modelling and optimisation of the beamlines under construction.
My main activity is the conception, development, maintenance, documentation and training of
computer codes for scientific applications. I participated in the optics design of most ESRF beamlines. I actively participated in the coding and
documentation of the SHADOW package. I am its main maitainer and developer today. I developed at ESRF a visual user interface, new algorithms and
created educational material (courses, tutorials, workshops).

I was also involved in the development of data analysis tools targeted for different beamlines and
synchrotron techniques (X-ray absorption spectroscopy, X-ray fluorescence, etc.). I also used a
Monte Carlo radiation transport package for simulation on a multi-strip solid state detector and for
the calculation of dose deposition in Microbeam Radiation Therapy.

\entry*[]
I have collaborated with scientists developing crystal optics for thermal neutron instrumentation (ILL, France), refractive lenses for synchrotron beamlines (Diamond Light Source, UK), crystal and multilayers optics for
astrophysics and X-ray telescopes (Universit\'a di Ferrara and Osservatorio Astronomico di Brera, Italy), and crystal spectrometers for the diagnostics of plasmas
(Multicharged Ions Spectra Data Center, Russia; Plasma Science and Fusion Center at the Massachusetts Institute of Technology and Plasma Physics Laboratory at Princeton University). 

In the last years I worked in modelling X-ray instrumentation for new synchrotron sources (low emittance storage rings). I developed new methods for wave-optics and hybrid ray-tracing. I was the first to apply the coherent mode decomposition to the undulator radiation, and modeled the partial coherence of these sources. In collaboration with APS at Argonne National Laboratory I developed the OASYS toolbox. It is a multi-purpose user-interface that lumps together many packages and models for optics simulations. It has been applied to the design and optimization of the beamlines at EBS-ESRF upgrade and in many other laboratories and synchrotron sources throughout the world.

I have always conducted an experimental activity in parallel with computer
modelling. I proposed and run several experiments at ESRF and ILL on characterization and performances of optics, but also for studies in mineralogy and cultural heritage. I am interested in investigating technical and scientific developments of ancient cultures. In 2005 I joined the Universidad Aut\'noma Metropolitana at Mexico City as professor in the Physics depertment. I studied the Maya blue pigment, a fascinating achievement of Mayan technology. 
%I worked in infrared optics, contributed to simulate a Raman Spectrometer. 
I was also interested in the mirrors used by the early cultures. I developed a project for tracking the provenance of obsidian used by the Teotihuacans, a material used for mirrors as metals were not known at that time.

In 2019 I was appointed at the Lawrence Berkeley National Laboratory to participate in the the upgrade of the Advanced Light Source. I was involved in  developments for future beamlines, including modeling new optics (e.g. adaptive optics or diaboloid mirrors).


% Since September 1990, I am staff member at the Experiments Division of the European
% Synchrotron Radiation Facility. All my professional activity is devoted to synchrotron radiation
% instrumentation and research.

% During my formation period (Ph.D. thesis) I worked at the Frascati (LNF-INFN) and Wisconsin
% (CXrL, UW) synchrotrons. Then I joined the ESRF in 1990 where I participated in the support
% groups (computing) to the design, construction and operation of the beamlines.
% My main activity is the conception, development, maintenance, documentation and training on
% computer codes for scientific applications (speci% collaboration (Mexico-France-Spain-Cuba) for applications in cultural heritage. I am currently
% studying the Maya blue pigment.ally modelling and simulation of physical
% phenomena, visualization, data analysis and fitting of experimental data). I give computational
% support for the optical modelling and optimisation of the x-ray optics. I participated in the design
% of several ESRF beamlines and performed calculations for the optics of most ESRF beamlines.
% I extensively used the ray-tracing code SHADOW. I actively participated in the coding and
% documentation of this package. I developed at ESRF a visual user interface, new algorithms and
% created educational material (courses, tutorials, workshops). These tools are used worldwide
% today.

% The computational needs for optics calculations at ESRF required not only ray-tracing simulation,
% but also a toolkit for performing quick calculations in an easy way, that could be used in an easy
% way at the beamlines. This was materialized in the package XOP, now a standard tool in most
% synchrotron facilities and in other laboratories, with more than 400 registered users.
% I was also involved in the development of data analysis tools targeted for different beamlines and
% synchrotron techniques (x-ray absorption spectroscopy, x-ray fluorescence, etc.). I also used a
% Monte Carlo radiation transport package for simulation on a multi-strip solid state detector and for
% the calculation of dose deposition in Microbeam Radiation Therapy.
% The expertise I obtained in the synchrotron field has been extended to other fields, and I have
% fruitful collaborations with scientists working with thermal neutron instrumentation (ILL, France),
% x-ray telescopes for astrophysics (Osservatorio Astronomico di Brera, Italy), and x-ray plasma
% sources (Multicharged Ions Spectra Data Center, Russia).
% I always wanted to encourage and maintain an experimental activity in parallel with the computer
% modelling, and I proposed and run several experiments at ESRF and ILL.
% I am also interested in expanding the applicability of SR to new research areas. I have started a
% collaboration (Mexico-France-Spain-Cuba) for applications in cultural heritage. I am currently
% studying the Maya blue pigment.

\end{rubric}