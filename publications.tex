%% Sometimes when a section can't be nicely modelled with the \entry[]... mechanism; hack our own
\makerubrichead{Research Publications}

%% Assuming you've already given \addbibresource{own-bib.bib} in the main doc. Right? Right???
\nocite{*}


I have over 150 research papers in peer-reviewed international research journals:
\begin{itemize}
    \item The full list of publications is found at \makefield{\faLink}{\url{http://publicationslist.org/srio}}
    \item 119
    %160-41
    Journal publications, 42
    % 54-12
    as first author
    \item plus 41 SPIE publications, 12 as first author
    % \item XX Book chapters, XX as first author
    % \item XX Conference presentations
    \item Citations: 3996  •  h-index: 34  •  i10-index: 81  •  Google Scholar (as of May. 16, 2021)



\end{itemize}



% For all of MSR’s works: XX have 100 citations or more, XX have 200 or more, XX have 500 or more and XX has 1000 or more.
\inred{A list of selected papers follows. }


% If you just want everything in one list
\printbibliography[heading={none}]
% \printbibliography[heading={none},keyword={selected}]


% \printbibliography[heading={subbibliography},title={Journal Articles},type=article]

% \printbibliography[heading={subbibliography},title={Conference Proceedings},type=inproceedings]

% \printbibliography[heading={subbibliography},title={Books and Chapters},filter={booksandchapters}]

% \printbibliography[heading=subbibliography]