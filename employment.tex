\begin{rubric}{Employment History}
% \entry*[]
%     My professional activity is devoted to X-ray instrumentation modelling and research using synchrotron light. It involves three main fields:
%     \begin{itemize}
%         \item Design, development, and use of computer codes in particular ray-tracing techniques for modelling X-ray optics with direct applications to the design of synchrotron radiation beamlines.
%         \item Development and use of data analysis tools 
%     specific for different synchrotron techniques (X-ray absorption spectroscopy, X-ray diffraction, X-ray fluorescence, etc.).
%         \item Experimental activity study of cultural heritage. Several projects concern the structure and chemistry of the Maya blue pigment.
%     \end{itemize}
\entry*[]% 
\textbf{Since 1990 I am staff Physicist at the European Synchrotron Radiation Facility (Grenoble, France). My professional activity at the ESRF is devoted to synchrotron radiation instrumentation and research. \inred{It concerns modeling and simulation of X-ray optical systems. My contributions in this area are exported and used worldwide, so I am recognised as one of the major specialists in this area.} I was on leave from ESRF in two occasions. I worked in 2005 at the Universidad Aut\'onoma Metropolitana (Mexico) on cultural heritage. In 2019-2020 I was employed by the Lawrence Berkeley National Laboratory (Berkeley, CA, USA) to work in the upgrade of the ALS storage ring.}

\entry*[2020/6/14 -- $\cdots\cdot$]% 
Engineer at the Advanced Analysis and Precision Unit at the Mechanical Engineering Group, Instrument Support and Development Division, ESRF.   

\entry*[2019/7/15 –- 2020/6/14]
Project Engineer at the Advanced Light Source, Lawrence Berkeley National Laboratory, Berkeley, CA, USA. 

\entry*[2009 –- 2019]  Computational Physicist at the Advanced Analysis and Modelling Unit at the Instrument Support and Development Division, ESRF. 

% Responsibilities: 
% \begin{itemize}
%     \item Providing scientific and engineering expertise to the ESRF for development, maintenance, and use of modelling tools for beamline optics. 
%   \item Accomplishments: 
%   \begin{itemize}
%   \item Direct participation in the optics design of several beamlines for the ESRF Upgrade Programme.
%   \item Promote international collaborations on X-ray optics software.
%   \item Development of the Oasys simulation software suite. 
%   \end{itemize}
%   \item Advise Ph.D. students. 
% \end{itemize}

\entry*[2009 –- 2006]  Deputy Head of the Scientific Software Unit at the Technical Beamline Support Group, Experiments Division, ESRF.

% Responsibilities: 
% \begin{itemize}
% \item Coordination and development of software for the analysis of experimental data. 
% \item Maintenance of optics simulation tools and advising.
% \end{itemize}
% Accomplishments: 
% \begin{itemize}
% \item Development of data analysis tools for several techniques (e.g., X-ray diffraction and fluorescence). 

\entry*[2005] Professor (chair Juan B. Oyarz\'abal) at the Physics Department, Universidad Aut\'onoma Metropolitana, Mexico City, Mexico.

% Responsibilities: 
% \begin{itemize} 
% \item Research on cultural heritage (Maya blue pigment). 
% \item Teach post-graduate courses.
% \end{itemize}
% Accomplishments: 
% \begin{itemize} 
% \item  Experimental results using RAMAN, FTIR, PIXE, SEM, NRM on ancient materials. 
% \item Seminars on archaeometry for scientists, archaeologists, and historians. 
% \end{itemize}

\entry*[2000 –- 2004]  Deputy Head of the Scientific Software Unit at the Technical Beamline Support Group, Experiments Division, ESRF.

% Responsibilities: 
% \begin{itemize} 
% \item Coordination and development of software for data analysis and optics simulations. 
% \item Maintenance of optics simulation tools and advising. 
% \end{itemize}
% Accomplishments:
% \begin{itemize} 
% \item Development of XOP, that became a standard tool in most synchrotron facilities with more than 400 users. 
% \item Advise a Ph.D. and several trainee students.
% \end{itemize}

\entry*[1993 –- 1999] 	Software Engineer in the Programming Group, Experiments Division, ESRF.

% Responsibilities: 
% \begin{itemize} 
% \item Perform X-ray optics simulations for the commissioning and optimization of ESRF beamlines.
% \end{itemize}
% Accomplishments: 
% \begin{itemize}
% \item Implement a computer infrastructure for ray-tracing simulations. 
% \end{itemize}

\entry*[1990 –- 1992]	Optics Scientist at the Experiment Division, ESRF.

% Responsabilities:
% \begin{itemize} 
% \item Perform X-ray optics simulations for the ESRF beamlines in the construction phase.
% \end{itemize}
% Accomplishments:
% \begin{itemize} 
% \item Participate in the optics design of several novel beamlines.
% \item Define a computer infrastructure for ray-tracing simulations. 
% \end{itemize}

\end{rubric}